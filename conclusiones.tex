Este trabajo presentó una caracterización experimental exhaustiva del efecto de la obstrucción corporal en un sistema de posicionamiento UWB operando a 6.5 GHz. Se ha demostrado cuantitativamente que la ubicación del dispositivo en el cuerpo es un factor crítico que puede degradar la exactitud de la estimación de distancia en más de un orden de magnitud.

Las principales conclusiones son:
\begin{enumerate}
    \item \textbf{La ubicación en la cabeza ofrece la mayor robustez y consistencia} frente a la obstrucción corporal, mientras que la cadera y el pecho son las más susceptibles a errores severos en condiciones NLOS. La muñeca se presenta como una alternativa práctica y equilibrada.
    \item \textbf{El error inducido por la BS en NLOS no es gaussiano}, lo que exige el uso de algoritmos de localización y filtrado que sean robustos a valores atípicos y distribuciones asimétricas.
    \item \textbf{La banda de 6.5 GHz es viable para el seguimiento de personas}, alcanzando una exactitud de posicionamiento 2D del orden de 55-95 cm en condiciones realistas, lo cual es suficiente para una amplia gama de aplicaciones.
    \item \textbf{El multitrayecto en interiores puede tener un efecto constructivo}, mejorando la exactitud en NLOS para ciertas ubicaciones en comparación con escenarios de exteriores sin reflexiones.
\end{enumerate}

Las contribuciones de esta investigación incluyen una base de datos experimental para la banda de 6.5 GHz, guías de diseño prácticas para sistemas comerciales, y benchmarks de rendimiento que pueden servir como referencia para futuros trabajos.

Como trabajo futuro, se recomienda extender el análisis a un mayor número de participantes para modelar la variabilidad antropométrica, evaluar escenarios más complejos y dinámicos, y desarrollar algoritmos de mitigación avanzados que fusionen datos de UWB con sensores inerciales para mejorar la robustez y precisión del sistema.
