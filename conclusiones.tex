Este trabajo presentó una caracterización experimental exhaustiva del efecto de la obstrucción corporal en un sistema de posicionamiento UWB operando a 6.5 GHz. Se ha demostrado cuantitativamente que la ubicación del dispositivo en el cuerpo es un factor crítico que puede degradar la exactitud de la estimación de distancia en más de un orden de magnitud.

Las principales conclusiones de este estudio indican que la ubicación en la cabeza proporciona la mayor robustez y consistencia frente a la obstrucción corporal, mientras que la cadera y el pecho resultan ser las posiciones más vulnerables a errores significativos en condiciones de falta de línea de vista. Asimismo, se determinó que el error inducido por la obstrucción corporal en condiciones NLOS no sigue una distribución gaussiana, lo que resalta la necesidad de emplear algoritmos de localización y filtrado capaces de gestionar valores atípicos. Por otro lado, los resultados confirman que la banda de 6.5 GHz es plenamente viable para el seguimiento de personas, logrando una exactitud de posicionamiento 2D de entre 55 y 95 cm. Finalmente, se observó que el multitrayecto en interiores puede generar un efecto constructivo que mejora la precisión en condiciones de obstrucción en comparación con entornos exteriores despejados.

Las contribuciones de esta investigación incluyen una base de datos experimental para la banda de 6.5 GHz, guías de diseño prácticas para sistemas comerciales, y benchmarks de rendimiento que pueden servir como referencia para futuros trabajos.

Como trabajo futuro, se recomienda extender el análisis a un mayor número de participantes para modelar la variabilidad antropométrica, evaluar escenarios más complejos y dinámicos, y desarrollar algoritmos de mitigación avanzados que fusionen datos de UWB con sensores inerciales para mejorar la robustez y precisión del sistema.
