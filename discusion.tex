\subsection{Impacto de la Ubicación Corporal en la Exactitud UWB}

Los resultados demuestran que la ubicación del dispositivo es el factor determinante primario en la exactitud del sistema. La cabeza ofrece consistencia superior debido a que la geometría del cráneo facilita la difracción de ondas de 6.5 GHz alrededor de su superficie curva, resultando en retardos predecibles y moderados incluso en NLOS (factor de degradación 2.0x en interiores). Por el contrario, el torso (cadera, pecho) sufre atenuación severa (factores de degradación 10x+), explicándose por el efecto combinado de: (i) sección transversal grande perpendicular a la dirección de propagación, (ii) contenido de agua y electrolitos que causan pérdida dieléctrica significativa a 6.5 GHz, (iii) geometría que crea zonas de sombra completa cuando el sujeto se interpone entre el tag y los anclajes. Estos hallazgos cuantifican sistemáticamente en la banda de 6.5 GHz (menos estudiada que 3--5 GHz) un fenómeno bien documentado en frecuencias inferiores \cite{ref4, ref16, ref14, ref15}.

\subsection{No-Gaussianidad del Error y Sesgo Sistemático}

La naturaleza no-Gaussiana del error en NLOS es una conclusión fundamental con implicaciones críticas para algoritmos de filtrado. Los métodos clásicos de localización (trilateración por mínimos cuadrados, Kalman lineal) se fundamentan en la asunción de ruido de medición gaussiano con media cero: $p(v_k) = \mathcal{N}(0, R)$. Nuestros análisis estadísticos (Tabla 3.2: distribución Log-normal óptima; Figura 3.5: test Shapiro-Wilk $p \ll 0.001$) demuestran que esta asunción es VIOLADA bajo obstrucción corporal severa.

Más críticamente aún, la obstrucción corporal no solo introduce ruido de varianza elevada, sino que introduce \textbf{sesgo positivo sistemático}: $d_{\text{medida}} = d_{\text{real}} + \text{bias}(\text{ubicación})$ donde el bias es función determinística de la ubicación corporal y la geometría. Este sesgo es inherente a la física del shadowing: la señal multitrayecto es retardada respecto a la LOS, por lo que las mediciones de ToF son consistentemente mayores que la distancia euclidiana real. La trilateración simple falla dramáticamente bajo este régimen porque minimiza la suma de residuos respecto a las mediciones sesgadas, localizando la posición estimada lejos de la verdadera.

\subsection{Análisis Riguroso del Desempeño del Filtro de Kalman Extendido}

En el contexto de estos errores, el desempeño del Filtro de Kalman Extendido (EKF) empleado en Fase 2 requiere análisis cauteloso. Si bien el EKF es teóricamente óptimo para sistemas no lineales con ruido gaussiano, su aplicación en nuestro escenario de obstrucción corporal severa revela limitaciones fundamentales.

\textbf{Desempeño en Caso de Referencia (sin cuerpo):} Cuando el tag fue portado sin cuerpo humano (Figura 3.4), el EKF converge eficazmente a las posiciones verdaderas, logrando error de posicionamiento promedio de 48 cm. En este régimen, las asunciones del filtro son válidas: (i) el ruido es predominantemente gaussiano, (ii) la media es cercana a cero (pequeños sesgos de calibración), (iii) la dinámica es lineal (modelo de velocidad constante captura adecuadamente el movimiento del tag entre puntos). El filtro suaviza eficazmente el jitter (reducción del 41\% en variabilidad temporal) y estima la trayectoria correctamente.

\textbf{Degradación Severa en Cadera NLOS:} Al introducir obstrucción corporal en la cadera, surge un problema de \textbf{desajuste de modelos} (model mismatch). El EKF sigue asumiendo $\mathbf{v}_k \sim \mathcal{N}(0, \mathbf{R})$ pero recibe mediciones con sesgo positivo constante (Tabla 3.1: bias promedio de ~90 cm). El filtro no puede distinguir entre ``ruido legitimamente gaussiano'' y ``sesgo sistemático'', por lo que su mecanismo de actualización converge al estimado que minimiza el error cuadrático respecto a las mediciones sesgadas. El resultado es una trayectoria suavizada que traza fielmente la geometría del sesgo, no la geometría verdadera. En la Figura 3.9, es evidente que: (i) la línea azul (EKF) es mucho más suave que los puntos naranjas (multilateración cruda), lo que indica que el filtro ESTÁ FUNCIONANDO (reducción de jitter), pero (ii) la línea azul está consistentemente desplazada de los puntos verdes (posiciones verdaderas) por decenas de centímetros, lo que demuestra que el filtro converge al estimado INCORRECTO.

Matemáticamente, el filtro soluciona: $\min_{\mathbf{x}} \sum_{k} \|\mathbf{z}_k - h(\mathbf{x}_k)\|^2_{\mathbf{R}^{-1}}$ donde $\mathbf{z}_k = d_{\text{real}} + \text{bias} + \eta_k$ (mediciones con sesgo más ruido). Sin acceso a información que le permita estimar separadamente el $\text{bias}$, el filtro minimiza la norma incluyendo el sesgo en su estado estimado, resultando en localización incorrecta.

\textbf{Contraste con Cabeza:} Por el contrario, cuando el tag está en la cabeza (predicho por los resultados de Fase 1: MAE = 18.16 cm interior NLOS), el sesgo es mucho menor (~18 cm frente a ~90 cm en cadera). En este régimen, el error es más cercano a la asunción de media cero del EKF, y el filtro convergería mucho más cerca de la posición verdadera (predicción: error de posicionamiento 2D en rango de 25--40 cm para cabeza, comparado con 55--95 cm para cadera).

\subsection{Implicaciones para Aplicaciones Prácticas}

Estos hallazgos establecen un límite fundamental de rendimiento para sistemas UWB sin técnicas de mitigación de NLOS: \textbf{el EKF estándar es un suavizador de trayectoria, no un corrector de sesgo}. Para ubicaciones corporales con LOS predominante o buenos ángulos de propagación (cabeza, muñeca), el sistema presente es robusto y suficiente para aplicaciones de nivel de habitación. Sin embargo, para ubicaciones con shadowing severo (cadera, pecho), la exactitud de 55--95 cm es insuficiente para aplicaciones de precisión e inalcanzable mediante filtrado temporal únicamente.

La solución requeriría técnicas de:
\begin{enumerate}
\item \textbf{Detección de NLOS:} Identificar en tiempo real cuáles mediciones de distancia tienen sesgo significativo, mediante indicadores como: variabilidad temporal elevada (Tabla 3.1: $\sigma = 0.63$ ns en NLOS vs 0.18 ns en LOS), distribución de colas pesadas (Figura 3.5: Q-Q plot), o indicadores de potencia de recepción.
\item \textbf{Corrección de Sesgo (Bias Correction):} Estimar y restar el sesgo antes de la localización, posiblemente usando modelos empíricos o aprendizaje automático que relacionen potencia de señal con sesgo esperado.
\item \textbf{Filtros Robustos:} Reemplazar el EKF con filtros que no asuman media cero (ej: filtro de Kalman aumentado con parámetro de bias, o filtros M-estimadores robustos a outliers).
\end{enumerate}

Estas técnicas están fuera del alcance del presente trabajo pero representan avenidas prioritarias para investigación futura.

\subsection{Multitrayecto Constructivo en Interiores}

Un resultado secundario pero notable fue el comportamiento diferencial entre escenarios exterior e interior para extremidades (tobillo, mano). En exteriores, la señal NLOS se pierde severamente. En el pasillo interior (paredes cercanas), la presencia de superficies reflectantes proporciona trayectorias alternativas que, aunque introducen retardos, mantienen conectividad de señal y evitan pérdida completa. Este fenómeno no es una ``mejora'' de precisión sino un cambio cualitativo en la naturaleza del error: de ``señal débil con ruido alto'' a ``múltiples trayectorias con retardos consistentes''. Este hallazgo es relevante para el diseño de sistemas UWB en espacios confinados (almacenes, hospitales, fábricas).

\subsection{Banda de 6.5 GHz como Lacuna de Investigación}

Este trabajo llena una brecha importante: mientras que abundan estudios a 3--5 GHz (banda industria UWB estándar), la banda de 6.5 GHz (caracterizada por mejor resolución temporal pero mayor atenuación por tejido) había sido poco estudiada en el contexto específico de shadowing corporal. Nuestros hallazgos cuantitativos (degradación 10x+ en cadera, distribución Log-normal universal en NLOS) proporcionan una base empírica para el diseño de sistemas futuros en esta banda.

\subsection{Limitaciones del Estudio}

Algunas limitaciones merecen mención:
\begin{enumerate}
\item Sujeto único en Fase 2: diferencias antropomórficas (masa corporal, composición) pueden afectar resultados.
\item Escenario de salón específico: geometría simplificada puede no representar ambientes complejos con múltiples reflejantes irregulares.
\item EKF sin adaptación: el filtro empleó matriz de covarianza $\mathbf{R}$ fija. Filtros adaptativos podrían mejorar desempeño.
\item Ausencia de estimación de sesgo: no se intentó técnicas de detección/corrección de NLOS, que podrían mejorar significativamente la exactitud.
\end{enumerate}
