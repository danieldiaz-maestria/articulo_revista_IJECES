Los resultados confirman que la ubicación del dispositivo es un factor determinante en la exactitud de los sistemas UWB. La consistencia superior de la \textbf{cabeza} como punto de montaje se atribuye a la difracción efectiva de las señales de 6.5 GHz alrededor de una superficie curva, en contraste con la alta atenuación que sufre la señal al intentar atravesar el torso, lo que explica el pobre desempeño de la \textbf{cadera} y el \textbf{pecho} en condiciones NLOS. Estos hallazgos son consistentes con estudios previos, pero este trabajo cuantifica el efecto de forma sistemática en la banda de 6.5 GHz para un amplio conjunto de ubicaciones.

La naturaleza no gaussiana de los errores en NLOS es una conclusión fundamental con implicaciones directas para el diseño de algoritmos. Los métodos de localización basados en mínimos cuadrados, que asumen ruido gaussiano, son subóptimos en presencia de BS. Esto justifica la necesidad de filtros más robustos, como el Filtro de Kalman, o algoritmos que puedan identificar y descartar mediciones NLOS. Aunque en nuestra fase 2 el Filtro de Kalman no siempre mejoró el error promedio, sí logró reducir significativamente los errores máximos (hasta un 49\%) y la variabilidad (hasta un 41\%), lo cual es crítico para la fiabilidad del sistema.

El fenómeno de multitrayecto constructivo en interiores para las extremidades es un resultado interesante. Mientras que en exteriores una señal NLOS se pierde en gran medida, en un corredor las reflexiones proveen caminos alternativos para la señal, mejorando la medición. Esto sugiere que, en ciertos entornos, el multitrayecto puede ser beneficioso y no solo una fuente de error.

La exactitud de posicionamiento 2D, entre 55 y 95 cm, es adecuada para muchas aplicaciones prácticas como el seguimiento de personal en logística, seguridad industrial o monitoreo de pacientes. Sin embargo, para tareas que requieren mayor precisión, como el análisis biomecánico detallado o la robótica, esta exactitud es insuficiente y se requerirían técnicas de mitigación adicionales.
