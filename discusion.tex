\subsection{Impacto de la Ubicación Corporal y Geometría de Propagación}

Los resultados obtenidos confirman que la ubicación del dispositivo en el cuerpo es un factor determinante en la precisión del sistema de posicionamiento UWB a 6.5 GHz. La ubicación en la cabeza demostró una robustez superior, atribuible a su geometría convexa que facilita la difracción de las ondas alrededor de la superficie, manteniendo la integridad de la señal y permitiendo retardos predecibles incluso en condiciones sin línea de visión (NLOS). En contraste, las ubicaciones en el torso (pecho y cadera) sufrieron degradaciones severas, con errores que aumentaron significativamente (factores de degradación superiores a 10x). Esto se explica por la gran sección transversal del torso y la composición biológica rica en agua, que bloquean efectivamente la señal directa, forzando la medición a través de trayectorias reflejadas más largas o atenuadas. Estos hallazgos corroboran y cuantifican para la banda de 6.5 GHz fenómenos previamente observados en frecuencias inferiores \cite{ref4, ref16, ref14, ref15}.

Adicionalmente, se observó un comportamiento diferencial notable donde ciertas ubicaciones presentaron una conectividad más robusta en interiores que en exteriores bajo condiciones NLOS. Esto sugiere que, en ambientes confinados, la presencia de múltiples superficies reflectantes facilita un "multitrayecto constructivo". Aunque la señal directa está bloqueada por el cuerpo, las reflexiones en paredes y techo permiten mantener la disponibilidad del servicio, a diferencia del escenario exterior donde la ausencia de reflectores cercanos resulta en la pérdida total de la señal. Este hallazgo es relevante para el diseño de sistemas UWB en entornos industriales densos.

\subsection{Naturaleza del Error y Desempeño del Filtrado}

Un hallazgo crítico de este estudio es la naturaleza no gaussiana y sesgada del error en condiciones de bloqueo corporal severo. Los algoritmos de localización estándar, como el Filtro de Kalman Extendido (EKF), asumen implícitamente que el ruido de medición sigue una distribución normal con media cero ($\mathcal{N}(0, R)$). Sin embargo, nuestros análisis estadísticos demuestran que el bloqueo corporal introduce un sesgo positivo sistemático (la distancia medida es consistentemente mayor a la real debido al retardo por multicamino) y una distribución de error que se ajusta mejor a modelos Log-normales.

El análisis del desempeño del EKF reveló que, en ausencia de obstrucción o con obstrucción leve (cabeza), el filtro opera correctamente suavizando el ruido aleatorio. No obstante, en escenarios de obstrucción severa (cadera), el filtro converge hacia una trayectoria incorrecta pero suave. Esto se debe a un desajuste de modelo (\textit{model mismatch}): el filtro interpreta el sesgo sistemático como la posición verdadera, ya que carece de información a priori para distinguir entre un desplazamiento real y un error de medición constante. Esto demuestra que el EKF estándar actúa eficazmente como un suavizador de trayectoria, pero es ineficaz para corregir errores de sesgo por NLOS sin mecanismos auxiliares.

\subsection{Implicaciones y Limitaciones}

Para aplicaciones prácticas, estos resultados implican que la elección de la ubicación del \textit{tag} es una decisión de diseño crítica. Para aplicaciones que requieren alta precisión y robustez, la cabeza o ubicaciones altas en el cuerpo son preferibles. Si el dispositivo debe portarse en el torso, es imperativo implementar algoritmos de mitigación de NLOS que puedan detectar condiciones de bloqueo mediante indicadores como la varianza del tiempo de vuelo o la potencia de señal, y compensar el sesgo antes de la etapa de filtrado.

Finalmente, es importante mencionar las limitaciones del estudio, como el uso de un único sujeto de prueba y un escenario geométrico simplificado. Futuras investigaciones deberían abordar la recolección de datos con múltiples sujetos para generalizar los efectos antropométricos y explorar técnicas adaptativas de estimación de sesgo en tiempo real para mejorar la precisión en ubicaciones críticas como la cadera.
