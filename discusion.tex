Los resultados confirman que la ubicación del dispositivo es un factor determinante en la exactitud de los sistemas UWB. La consistencia superior de la \textbf{cabeza} como punto de montaje se atribuye a la difracción efectiva de las señales de 6.5 GHz alrededor de una superficie curva, en contraste con la alta atenuación que sufre la señal al intentar atravesar el torso, lo que explica el pobre desempeño de la \textbf{cadera} y el \textbf{pecho} en condiciones NLOS. Estos hallazgos son consistentes con estudios previos que han identificado al cuerpo como un obstáculo significativo en frecuencias inferiores \cite{ref4, ref16, ref14, ref15}, pero este trabajo cuantifica el efecto de forma sistemática en la banda de 6.5 GHz para un amplio conjunto de ubicaciones.

La naturaleza no gaussiana de los errores en NLOS es una conclusión fundamental con implicaciones directas para el diseño de algoritmos \cite{ref11, ref18}. Los métodos de localización clásicos basados en mínimos cuadrados asumen que el error de medición sigue una distribución normal con media cero. Sin embargo, nuestros datos demuestran que la obstrucción corporal introduce un sesgo positivo significativo. Esta violación de las asunciones estadísticas básicas explica por qué la multilateración simple falla dramáticamente en ciertos puntos de la trayectoria experimental.

En este contexto, el desempeño del Filtro de Kalman merece un análisis detallado. Aunque teóricamente óptimo para sistemas lineales con ruido gaussiano, su aplicación directa en escenarios con fuerte obstrucción corporal (NLOS severo) presenta limitaciones. En nuestros resultados de la fase 2, observamos que si bien el filtro logró suavizar la trayectoria y reducir la variabilidad (jitter) hasta en un 41\%, no pudo corregir completamente el error de posición (bias) introducido por la obstrucción corporal constante, como se evidenció en los casos de cadera y pecho. Esto sugiere que para aplicaciones de alta precisión en 6.5 GHz, no basta con filtrar el ruido temporal; es necesario implementar mecanismos de detección y mitigación de NLOS previos a la etapa de estimación de posición que puedan integrar modelos no lineales del error inducido por el cuerpo.

El fenómeno de multitrayecto constructivo en interiores para las extremidades es un resultado interesante. Mientras que en exteriores una señal NLOS se pierde en gran medida, en un corredor las reflexiones proveen caminos alternativos para la señal, mejorando la medición. Esto sugiere que, en ciertos entornos, el multitrayecto puede ser beneficioso y no solo una fuente de error.

La exactitud de posicionamiento 2D, entre 55 y 95 cm, es adecuada para muchas aplicaciones prácticas como el seguimiento de personal en logística, seguridad industrial o monitoreo de pacientes. Sin embargo, para tareas que requieren mayor precisión, como el análisis biomecánico detallado o la robótica, esta exactitud es insuficiente y se requerirían técnicas de mitigación adicionales.
