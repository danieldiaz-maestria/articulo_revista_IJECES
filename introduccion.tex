Durante las últimas dos décadas, la interacción humana con el entorno tecnológico ha experimentado una evolución significativa, impulsada en gran medida por el desarrollo de los Sistemas de Posicionamiento en Interiores (IPS). En este contexto, la tecnología de Banda Ultra Ancha (UWB) ha destacado como una solución fundamental gracias a su capacidad para proporcionar una exactitud de localización subdecimétrica \cite{ref6, ref8}. A pesar de estas ventajas teóricas, su implementación en escenarios del mundo real enfrenta obstáculos críticos, entre los cuales la obstrucción corporal o \textit{Body Shadowing} (BS) es uno de los más complejos y determinantes \cite{ref16}.

El fenómeno de BS ocurre cuando el cuerpo humano se interpone entre el dispositivo móvil y los nodos de referencia fijos. Dado que el tejido biológico posee una alta permitividad debida a su contenido de agua, actúa como un medio que atenúa, refracta y difracta las señales de radiofrecuencia \cite{ref15, ref18}. Este efecto no es despreciable, ya que puede introducir errores en la estimación de distancia que degradan severamente la fiabilidad del sistema en aplicaciones críticas como la seguridad industrial \cite{Kianfar2020}, el monitoreo hospitalario o el análisis del rendimiento deportivo.

Estudios previos han cuantificado la magnitud de este problema, destacando su relevancia para sistemas de alta precisión. Investigaciones experimentales han reportado que la obstrucción corporal puede introducir errores de posicionamiento que oscilan entre 1.2 m y 4.5 m en condiciones de NLOS severo \cite{ref4}. Otros autores han documentado atenuaciones de señal de hasta 15 dB \cite{Ruonan2019} e incluso errores extremos de hasta 14 m en ángulos de incidencia particularmente desfavorables \cite{Schimtt2019}. Asimismo, comparativas de sistemas comerciales han evidenciado que, aunque plataformas líderes como Decaware (ahora Quorvo) pueden lograr error de exactitud de 12 cm en condiciones ideales, su rendimiento se degrada drásticamente sin algoritmos de mitigación adecuados, registrando errores superiores a 1 m en entornos complejos \cite{Falcone2012}. Estas cifras subrayan que la BS no es un error marginal, sino una barrera fundamental para la adopción de la tecnología.

Aunque la comunidad científica ha explorado diversas estrategias de mitigación, la mayoría de las investigaciones se han concentrado en el rango de frecuencias de 3 a 5 GHz \cite{ref4, ref14}. Sin embargo, la banda de 6.5 GHz ofrece un equilibrio interesante entre resolución temporal y características de propagación, a pesar de ser más sensible a la atenuación por obstruccion corporal \cite{ref20}. Existe, por tanto, una necesidad de caracterizar rigurosamente este fenómeno en dicha banda para comprender cómo la ubicación del dispositivo en el cuerpo influye en la exactitud del sistema.

El objetivo central de este artículo es analizar y cuantificar el impacto de la BS en un sistema UWB operando a 6.5 GHz. Este trabajo es una extensión y profundización de un estudio preliminar presentado previamente \cite{danny_daniel_and_victor_quintero}, en el cual se exploraron los efectos iniciales de la obstrucción corporal en la estimación de distancia en 7 posiciones coporales. En la presente investigación, se expande el análisis presentando un protocolo experimental que evalúa un sistema de posicionamiento en 2D con el dispositivo en la cadera y el pecho (ubicaciones corporales con la peor exactitud estadisticmaente hablando). El estudio no solo documenta la magnitud del error inducido por el cuerpo, sino que también propone la base para modelos estadísticos que puedan integrarse en algoritmos de localización más robustos, facilitando así que la tecnología UWB obtenga un mayor rendimiento en entornos más dinámicos y realistas.

El artículo se estructura de la siguiente manera: en la Sección \ref{sec:methodology} se detalla la metodología experimental empleada, incluyendo la configuración del hardware, los escenarios de prueba y el protocolo de recolección de datos. La Sección \ref{sec:results} presenta los resultados obtenidos, analizando el error de distancia por ubicación corporal y las implicaciones para el posicionamiento 2D. En la Sección \ref{sec:discussion}, se discuten los hallazgos principales, sus implicaciones prácticas y las limitaciones del estudio. Finalmente, la Sección \ref{sec:conclusions} resume las conclusiones clave y propone direcciones para futuras investigaciones en este campo.
