La tecnología de Banda Ultra Ancha (UWB) ha surgido como una solución clave para los Sistemas de Posicionamiento en Interiores (IPS), ofreciendo una exactitud subdecimétrica. Sin embargo, un desafío fundamental en aplicaciones reales es el fenómeno de la obstrucción corporal o \textit{Body Shadowing} (BS), donde el cuerpo humano atenúa y difracta las señales de radiofrecuencia, introduciendo errores significativos en la estimación de distancia y, por ende, en la posición final. Este problema es crítico para aplicaciones como la seguridad industrial, el seguimiento de activos en hospitales o el análisis de rendimiento deportivo.

La investigación existente se ha centrado mayoritariamente en frecuencias entre 3 y 5 GHz. En contraste, la banda de 6.5 GHz ha recibido menos atención, a pesar de ofrecer un compromiso potencialmente ventajoso entre resolución temporal y características de propagación. Este trabajo busca llenar ese vacío, proporcionando una caracterización experimental rigurosa del efecto de la BS en sistemas UWB que operan a 6.5 GHz.

El objetivo de este artículo es analizar y cuantificar el impacto de la ubicación del dispositivo en el cuerpo sobre la exactitud del sistema. Para ello, se ha diseñado un protocolo experimental que evalúa sistemáticamente siete ubicaciones corporales (cabeza, pecho, cadera, muñeca, mano, rodilla y tobillo) bajo condiciones controladas de Línea de Vista (LOS) y Sin Línea de Vista (NLOS). Los resultados no solo documentan la magnitud del error, sino que también proporcionan modelos estadísticos que pueden ser integrados en algoritmos de localización más robustos, contribuyendo así a que la tecnología UWB alcance su máximo potencial en aplicaciones del mundo real.
