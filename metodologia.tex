La investigación se basó en una metodología experimental dividida en dos fases: una primera fase de validación para evaluar la exactitud en la estimación de distancia punto a punto, y una segunda fase para evaluar el sistema de posicionamiento 2D completo.

\subsection{Configuración del Hardware y Escenario}
Se utilizaron módulos de desarrollo Qorvo DWM1001, configurados para operar en una frecuencia central de 6.5 GHz. El sistema experimental consistió en un nodo de referencia fijo y un nodo móvil portado por un sujeto de prueba.

Los experimentos se realizaron en dos escenarios:
\begin{enumerate}
    \item \textbf{Exterior:} Un campo abierto para minimizar reflexiones y aislar el impacto directo de la BS.
    \item \textbf{Interior:} Un corredor de edificio para introducir efectos de multitrayecto controlados, representativos de aplicaciones indoor.
\end{enumerate}

\subsection{Protocolo de Recolección de Datos}
Se evaluaron siete ubicaciones corporales del dispositivo móvil: cabeza, pecho, cadera, mano, muñeca, rodilla y tobillo. Para cada ubicación, se tomaron medidas en dos condiciones:
\begin{itemize}
    \item \textbf{LOS (Line of Sight):} Con una trayectoria visual directa entre el nodo fijo y el móvil.
    \item \textbf{NLOS (Non-Line of Sight):} Obstruyendo deliberadamente la señal con el cuerpo del sujeto (rotación de 180°).
\end{itemize}
Las mediciones se realizaron en 13 puntos estáticos, a distancias de 1 m a 13 m. En cada punto, se registraron 250 mediciones de distancia, acumulando un total de 91,000 mediciones en la primera fase.

\subsection{Análisis de Datos}
El error de distancia se calculó como $e_i = d_{\text{medida},i} - d_{\text{real},i}$. Se utilizaron métricas como el Error Absoluto Medio (MAE) y la Raíz del Error Cuadrático Medio (RMSE) para cuantificar el rendimiento. Dado que las distribuciones de error mostraron ser no gaussianas (prueba de Shapiro-Wilk, $p \ll 0.001$), se emplearon pruebas no paramétricas, como la U de Mann-Whitney, para el análisis inferencial.
