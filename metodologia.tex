La investigación se basó en una metodología experimental dividida en dos fases: una primera fase de validación para evaluar la exactitud en la estimación de distancia punto a punto, y una segunda fase para evaluar el sistema de posicionamiento 2D completo.

Para la configuración del hardware y el escenario, se utilizaron módulos de desarrollo Qorvo DWM1001, configurados para operar en una frecuencia central de 6.5 GHz \cite{ref8}. El sistema experimental consistió en un nodo de referencia fijo y un nodo móvil portado por un sujeto de prueba.

Los experimentos se realizaron en dos escenarios distintos: un campo abierto en el exterior, seleccionado para minimizar las reflexiones y aislar el impacto directo de la \textit{Body Shadowing}, y un corredor de edificio en el interior para introducir efectos de multitrayecto controlados que fueran representativos de las aplicaciones en espacios cerrados.

En relación con el protocolo de recolección de datos, se evaluaron siete ubicaciones corporales del dispositivo móvil: cabeza, pecho, cadera, mano, muñeca, rodilla y tobillo, tal como se ilustra en la Figura \ref{fig:posiciones}. Para cada ubicación, se tomaron medidas bajo dos condiciones de propagación principales. En primer lugar, la condición LOS (\textit{Line of Sight}) con una trayectoria visual directa entre el nodo fijo y el móvil. En segundo lugar, la condición NLOS (\textit{Non-Line of Sight}), donde se obstruyó deliberadamente la señal rotando el cuerpo del sujeto 180°.

\begin{figure}[H]
    \centering
    \includegraphics[width=0.8\linewidth]{posiciones_cuerpo.jpg}
    \caption{Ubicación de los nodos en el cuerpo para las pruebas experimentales.}
    \label{fig:posiciones}
\end{figure}
Las mediciones se realizaron en 13 puntos estáticos, a distancias de 1 m a 13 m. En cada punto, se registraron 250 mediciones de distancia, acumulando un total de 91,000 mediciones en la primera fase.

Para la segunda fase, orientada a la evaluación del sistema de posicionamiento 2D, se habilitó un escenario experimental en un salón de 7.4 m $\times$ 10.4 m. Se desplegaron cuatro anclas fijas instaladas a una altura de 1.52 m, formando un rectángulo que cubre el área de interés (ver Figura \ref{fig:croquis}). Las coordenadas $(x, y)$ de las anclas, en metros, se establecieron de la siguiente manera: $A_1(10.11, 0.37)$, $A_2(0.28, 0.38)$, $A_3(0.28, 6.72)$ y $A_4(10.09, 6.97)$.

\begin{figure}[H]
    \centering
    \includegraphics[width=0.8\linewidth]{RepresentacionPosiciones.jpeg}
    \caption{Croquis del escenario experimental 2D: Distribución de anclas y trayectoria de prueba.}
    \label{fig:croquis}
\end{figure}

El protocolo de prueba consistió en recorrer una trayectoria predefinida de 18 puntos distribuidos en una cuadrícula de 3 filas $\times$ 6 columnas, con una separación de 1.5 m entre puntos adyacentes. En esta fase se empleó el algoritmo de multilateración basado en Mínimos Cuadrados No Lineales para estimar la posición bruta, seguido de un Filtro de Kalman Extendido (EKF) para suavizar la trayectoria y mitigar el ruido de medición. El KF se configuró con un modelo de movimiento de velocidad constante, asumiendo una varianza del ruido de medición ajustada dinámicamente según la desviación estándar observada en la fase de caracterización.

Por último, para el análisis de datos, el error de distancia se calculó como $e_i = d_{\text{medida},i} - d_{\text{real},i}$. Se utilizaron métricas como el Error Absoluto Medio (MAE) y la Raíz del Error Cuadrático Medio (RMSE) para cuantificar el rendimiento. Dado que las distribuciones de error mostraron ser no gaussianas (prueba de Shapiro-Wilk, $p \ll 0.001$), se emplearon pruebas no paramétricas, como la U de Mann-Whitney, para el análisis inferencial.
