La investigación se basó en una metodología experimental de dos fases, cada una con propósitos específicos: Fase 1 (caracterización estática) para evaluar sistemáticamente el impacto de la BS sobre la estimación de distancia punto a punto en múltiples ubicaciones corporales; Fase 2 (validación dinámica) para evaluar el desempeño del sistema de posicionamiento 2D completo utilizando los modelos estadísticos derivados de Fase 1.

\textbf{Configuración del Hardware.} Los experimentos emplearon módulos de desarrollo Qorvo DWM1001, configurados para operar en una frecuencia central de 6.5 GHz \cite{ref8}. El sistema experimental consistió en cuatro nodos ancla fijos (Fase 2) o un nodo de referencia fijo (Fase 1) junto con un nodo móvil portado por participantes humanos.

\textbf{Escenarios Experimentales.} Las dos fases se ejecutaron en entornos distintos, elegidos deliberadamente para reflejar diferentes aspectos del problema:

\textit{Fase 1 (Caracterización de Distancia):} Se realizaron mediciones estáticas en dos escenarios complementarios. Primero, un campo abierto al aire libre (amplitud aproximada de 20~m $\times$ 30~m) seleccionado para minimizar reflexiones y aislar el impacto directo de la Body Shadowing. Segundo, un corredor interior (aproximadamente 2~m de ancho, 50~m de largo, con paredes y techos cercanos) para introducir efectos de multitrayecto controlados y representativos de aplicaciones en espacios cerrados. Esta dualizad de escenarios permitió diferenciar entre atenuación por obstrucción corporal pura (exterior) versus atenuación compuesta con reflexiones (interior).

\textit{Fase 2 (Posicionamiento Dinámico 2D):} Se utilizó un salón experimental de dimensiones 7.4~m $\times$ 10.4~m con cuatro anclas UWB fijas instaladas a una altura de 1.52~m. Este entorno representa un escenario de aplicación práctica (oficina, área de laboratorio) donde el sistema debe operar en condiciones realistas con multitrayecto y con los usuarios en movimiento.

\textbf{Fase 1: Protocolo de Caracterización de Distancia.} Se evaluaron siete ubicaciones corporales del nodo móvil: cabeza, pecho, cadera, mano, muñeca, rodilla y tobillo (Figura~\ref{fig:posiciones}). Para cada ubicación, se recopilaron medidas bajo dos condiciones de propagación: Línea de Vista (LOS) con trayectoria directa entre ancla y nodo móvil, y No-Línea de Vista (NLOS), donde se obstruyó deliberadamente la señal rotando el cuerpo del participante 180° para asegurar oclusión completa del nodo móvil desde el ancla.

Las mediciones se realizaron en 13 posiciones estáticas, a distancias de 1~m a 13~m desde el ancla de referencia. En cada posición, se registraron 250 mediciones independientes de distancia, acumulando aproximadamente 91,000 mediciones de distancia en total (13 posiciones $\times$ 7 ubicaciones corporales $\times$ 2 condiciones de canal $\times$ 250 muestras $\times$ 2 escenarios). Este volumen de datos permitió construcción de modelos estadísticos robustos para caracterizar la distribución de errores bajo cada condición.

\begin{figure}[H]
    \centering
    \includegraphics[width=0.8\linewidth]{posiciones_cuerpo.jpg}
    \caption{Ubicaciones de medición en el cuerpo humano durante Fase 1 de caracterización.}
    \label{fig:posiciones}
\end{figure}

\textbf{Análisis Estadístico de Fase 1.} El error de distancia se calculó como $e_i = d_{\text{medida},i} - d_{\text{real},i}$ para cada medición. Se ajustaron distribuciones de probabilidad teóricas (Gamma, Log-normal, Exponencial, Normal) a los datos empíricos utilizando Estimación de Máxima Verosimilitud (MLE). Se empleó el Criterio de Información de Akaike (AIC) para seleccionar el modelo más parsimonioso. La prueba de Shapiro--Wilk confirmó que los errores en NLOS se apartan significativamente de la normalidad ($p \ll 0.001$), validando la necesidad de distribuciones alternativas. Se calcularon las desviaciones estándar experimentales ($\sigma_{\text{LOS}}$ y $\sigma_{\text{NLOS}}$) para cada ubicación corporal bajo ambas condiciones, datos que fueron utilizados subsecuentemente para configurar la matriz de covarianza de ruido de medición del EKF en Fase 2.

\textbf{Fase 2: Validación de Posicionamiento 2D Dinámico.} El salón experimental de 7.4~m $\times$ 10.4~m fue equipado con cuatro anclas fijas cuyas coordenadas se determinaron mediante medición directa (Figura~\ref{fig:croquis}): $\mathbf{A}_1(10.11, 0.37)$, $\mathbf{A}_2(0.28, 0.38)$, $\mathbf{A}_3(0.28, 6.72)$, y $\mathbf{A}_4(10.09, 6.97)$ (en metros).

Un participante portó el nodo móvil en dos ubicaciones corporales---cadera y pecho, seleccionadas como casos extremos de desempeño (mejor y peor según Fase 1)---y se posicionó secuencialmente en cada uno de los 18 puntos de referencia distribuidos en una cuadrícula de 3 filas $\times$ 6 columnas (separación de 1.5~m entre puntos adyacentes). En cada punto, el participante se mantuvo estacionario durante aproximadamente 70 segundos (equivalente a $\sim$250 muestras) para registrar mediciones de distancia continuas desde los cuatro anclas.

\textbf{Algoritmo de Posicionamiento.} El sistema de posicionamiento implementó un Filtro de Kalman Extendido (EKF) diseñado para procesar directamente las mediciones de distancia (Time-of-Flight) provenientes de los cuatro anclas. A diferencia de enfoques que filtran coordenadas post-trilateración, este diseño fusiona las mediciones de rango crudas, permitiendo un manejo más robusto de la no linealidad geométrica.

El vector de estado del sistema se definió simplificadamente como la posición 2D del objetivo, $\mathbf{x}_k = [x_k, y_k]^\top$. Considerando la naturaleza del experimento (puntos estáticos o movimiento cuasi-estacionario), se empleó un modelo de transición de identidad con ruido de proceso aditivo para modelar la evolución temporal de la posición:
\begin{equation}
\mathbf{x}_{k+1} = \mathbf{x}_k + \mathbf{w}_k, \quad \mathbf{w}_k \sim \mathcal{N}(0, \mathbf{Q})
\end{equation}
donde $\mathbf{Q}$ es la matriz de covarianza del ruido del proceso, que permite al filtro adaptarse a los cambios de posición entre puntos de prueba.

La relación entre el estado y las mediciones es inherentemente no lineal. El vector de medición $\mathbf{z}_k$ contiene las distancias medidas a los cuatro anclas. La ecuación de observación se modela como:
\begin{equation}
z_{k,i} = h_i(\mathbf{x}_k) + v_{k,i} = \sqrt{(x_k - x_{A,i})^2 + (y_k - y_{A,i})^2} + v_{k,i}
\end{equation}
donde $(x_{A,i}, y_{A,i})$ son las coordenadas del $i$-ésimo ancla. En la etapa de corrección del EKF, esta función no lineal $h(\mathbf{x})$ se linealiza mediante el cálculo de la matriz Jacobiana $\mathbf{H}_k$ evaluada en la estimación de estado a priori:
\begin{equation}
\mathbf{H}_k = \left[ \begin{array}{cc}
\frac{x_k - x_{A,1}}{d_1} & \frac{y_k - y_{A,1}}{d_1} \\
\vdots & \vdots \\
\frac{x_k - x_{A,4}}{d_4} & \frac{y_k - y_{A,4}}{d_4}
\end{array} \right]
\end{equation}

Un componente crítico de la implementación es la definición dinámica de la matriz de covarianza de ruido de medición, $\mathbf{R}_k$. En lugar de asumir un error constante, el algoritmo utiliza las desviaciones estándar ($\sigma$) observadas en tiempo real para cada enlace UWB:
\begin{equation}
\mathbf{R}_k = \text{diag}(\sigma_{k,1}^2, \sigma_{k,2}^2, \sigma_{k,3}^2, \sigma_{k,4}^2)
\end{equation}
Esta configuración permite que el filtro pondere automáticamente la confianza de cada medición, otorgando menor peso a los enlaces que presentan alta variabilidad (típico de condiciones NLOS) y mayor peso a enlaces estables, mejorando así la estimación de la posición final.

\begin{figure}[H]
    \centering
    \includegraphics[width=0.8\linewidth]{RepresentacionPosiciones.jpeg}
    \caption{Croquis del escenario experimental Fase 2: distribución de anclas fijas y trayectoria de 18 puntos.}
    \label{fig:croquis}
\end{figure}

\textbf{Métricas de Desempeño.} Se utilizaron el Error Absoluto Medio (MAE) y Raíz del Error Cuadrático Medio (RMSE) para cuantificar la exactitud de posicionamiento. Dado que los errores de distancia exhiben distribuciones no gaussianas (como se confirmó en Fase 1), se emplearon complementariamente tests no paramétricos (U de Mann--Whitney) para validar diferencias estadísticas significativas entre ubicaciones corporales y condiciones de canal.
