Los resultados experimentales demuestran una fuerte dependencia entre la ubicación del dispositivo, el escenario y la condición de propagación (LOS/NLOS).

\subsection{Error de Distancia por Ubicación Corporal}
La Tabla \ref{tab:resumen_mae} consolida el Error Absoluto Medio (MAE) para todas las configuraciones evaluadas. De los datos se extraen los siguientes hallazgos clave:
\begin{itemize}
    \item \textbf{Mejor rendimiento general:} La \textbf{cabeza} ofrece el comportamiento más estable y predecible, con una degradación LOS-a-NLOS moderada, especialmente en interiores (factor de 2.0x). La geometría curva del cráneo parece facilitar la difracción de la señal.
    \item \textbf{Peor rendimiento general:} La \textbf{cadera} exhibe la degradación más severa en condiciones NLOS, con errores cercanos a 1 metro y factores de degradación superiores a 10x. El torso y la pelvis actúan como un obstáculo significativo.
    \item \textbf{Ubicaciones en extremidades:} La \textbf{muñeca} y la \textbf{mano} presentan un rendimiento moderado y consistente, siendo alternativas viables a la cabeza. El tobillo y la rodilla muestran errores más elevados debido a la proximidad con el suelo y la obstrucción de las piernas.
\end{itemize}

\begin{table}[h!]
\centering
\caption{Resumen de MAE (cm) por Ubicación y Condición.}
\label{tab:resumen_mae}
\begin{tabular}{lcccc}
\toprule
\multirow{2}{*}{\textbf{Ubicación}} & \multicolumn{2}{c}{\textbf{Exterior}} & \multicolumn{2}{c}{\textbf{Interior}} \\
\cmidrule(lr){2-3} \cmidrule(lr){4-5}
& \textbf{LOS} & \textbf{NLOS} & \textbf{LOS} & \textbf{NLOS} \\
\midrule
Cabeza   & 4.87 & 32.43 & 8.91 & 18.16 \\
Cadera   & 8.31 & 89.99 & 7.22 & 95.11 \\
Mano     & 9.28 & 42.17 & 9.09 & 26.01 \\
Muñeca   & 7.78 & 24.21 & 6.42 & 22.00 \\
Pecho    & 4.62 & 63.69 & 13.74 & 83.98 \\
Rodilla  & 14.99 & 51.36 & 12.04 & 49.52 \\
Tobillo  & 29.81 & 38.45 & 41.48 & 29.96 \\
\bottomrule
\end{tabular}
\end{table}

\subsection{Análisis Estadístico}
El análisis no paramétrico confirmó que las diferencias entre las configuraciones son estadísticamente significativas ($p \ll 0.001$) y de gran relevancia práctica. Un hallazgo clave fue el fenómeno de \textbf{multitrayecto constructivo} en el escenario interior: para varias ubicaciones (cabeza, mano, tobillo), el error en NLOS fue menor en interiores que en exteriores. Esto sugiere que las reflexiones en paredes y suelo pueden compensar parcialmente la atenuación causada por la obstrucción corporal.

\subsection{Rendimiento del Sistema de Posicionamiento 2D}
El sistema completo, con cuatro anclas, alcanzó una exactitud de posicionamiento promedio de:
\begin{itemize}
    \item \textbf{48 cm} para el tag de referencia (sin cuerpo).
    \item \textbf{55-95 cm} para el dispositivo en la cadera.
    \item \textbf{70-91 cm} para el dispositivo en el pecho.
\end{itemize}
Estos resultados validan la viabilidad del sistema para aplicaciones que requieren una localización a nivel de habitación, pero también resaltan la necesidad de técnicas de mitigación para alcanzar una precisión sub-métrica de manera consistente.
