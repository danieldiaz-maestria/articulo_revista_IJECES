Los resultados experimentales demuestran una fuerte dependencia entre la ubicación del dispositivo, el escenario y la condición de propagación (LOS/NLOS). 

En primer lugar, analizando el error de distancia por ubicación corporal, la Tabla \ref{tab:resumen_mae} consolida el Error Absoluto Medio (MAE) para todas las configuraciones evaluadas. El análisis de los datos revela que la cabeza ofrece el mejor rendimiento general, con un comportamiento estable y una degradación moderada entre LOS y NLOS (especialmente en interiores, con un factor de 2.0x), posiblemente debido a que la geometría del cráneo facilita la difracción. Por el contrario, la cadera presenta el peor rendimiento, con una degradación severa en NLOS donde los errores se acercan al metro y los factores de degradación superan 10x, confirmando que el torso actúa como un obstáculo crítico. Respecto a las extremidades, la muñeca y la mano muestran un rendimiento moderado y consistente que las posiciona como alternativas viables a la cabeza, mientras que el tobillo y la rodilla registran errores más elevados por la proximidad al suelo y la obstrucción de las piernas.

\begin{table}[H]
\centering
\caption{Resumen de MAE (cm) por Ubicación y Condición.}
\label{tab:resumen_mae}
\begin{tabular}{lcccc}
\toprule
\multirow{2}{*}{\textbf{Ubicación}} & \multicolumn{2}{c}{\textbf{Exterior}} & \multicolumn{2}{c}{\textbf{Interior}} \\
\cmidrule(lr){2-3} \cmidrule(lr){4-5}
& \textbf{LOS} & \textbf{NLOS} & \textbf{LOS} & \textbf{NLOS} \\
\midrule
Cabeza   & 4.87 & 32.43 & 8.91 & 18.16 \\
Cadera   & 8.31 & 89.99 & 7.22 & 95.11 \\
Mano     & 9.28 & 42.17 & 9.09 & 26.01 \\
Muñeca   & 7.78 & 24.21 & 6.42 & 22.00 \\
Pecho    & 4.62 & 63.69 & 13.74 & 83.98 \\
Rodilla  & 14.99 & 51.36 & 12.04 & 49.52 \\
Tobillo  & 29.81 & 38.45 & 41.48 & 29.96 \\
\bottomrule
\end{tabular}
\end{table}

Complementariamente, el análisis estadístico no paramétrico confirmó que las diferencias entre las configuraciones son estadísticamente significativas ($p \ll 0.001$). Un hallazgo relevante fue el comportamiento diferencial entre escenarios: para varias ubicaciones (cabeza, mano, tobillo), el error en NLOS fue menor en interiores respecto a exteriores. Este fenómeno no implica una ``mejora'' de la precisión, sino que refleja un cambio cualitativo en el mecanismo de propagación. En el entorno interior (pasillo con paredes cercanas), la proximidad de superficies reflectantes permite mantener la conectividad de la señal mediante trayectorias múltiples (reflexiones) que aunque introducen retardos adicionales y variabilidad temporal, garantizan la recepción continua de energía. En contraste, en el exterior, cuando el cuerpo obstruye el camino directo, la señal se atenúa severamente sin fuentes alternativas de reflexión cercana, resultando en pérdida de integridad de la señal \cite{ref11, ref17}. Esta distinción entre interior (multipath persistente) y exterior (ausencia de alternativas de propagación) es crítica para interpretar la naturaleza del error observado en Fase 2.

\subsection{Análisis Temporal del Tiempo de Vuelo (ToF)}
El análisis detallado del Tiempo de Vuelo (ToF) revela patrones críticos que explican tanto la degradación en la precisión de distancia como la estructura probabilística del error. La variabilidad temporal de la señal, cuantificada mediante la desviación estándar ($\sigma$), se identificó como el indicador más robusto para distinguir entre condiciones LOS y NLOS e informar la configuración de la matriz de covarianza del filtro de Kalman en Fase 2.

Como se observa en la Tabla \ref{tab:tof_stats}, al pasar de una condición de línea de vista a una de obstrucción corporal, la desviación estándar del ToF se incrementa drásticamente, con un aumento promedio de 3.5$\times$ (de 0.18~ns a 0.63~ns), ilustrando el aumento exponencial de la incertidumbre temporal bajo shadowing corporal. Este incremento de variabilidad es la manifestación temporal de los errores de distancia severos documentados en la Tabla \ref{tab:resumen_mae}.

\begin{table}[H]
\centering
\caption{Comparación estadística de ToF para ubicación en Pecho.}
\label{tab:tof_stats}
\resizebox{\columnwidth}{!}{%
\begin{tabular}{lccc}
\toprule
\textbf{Métrica} & \textbf{LOS (ns)} & \textbf{NLOS (ns)} & \textbf{Incremento} \\
\midrule
Desv. Std Promedio & 0.18 & 0.63 & 3.5x \\
Rango Máximo & 27.37 & 42.86 & 1.5x \\
\bottomrule
\end{tabular}%
}
\end{table}

La Figura \ref{fig:tof_pecho_dist} ilustra las distribuciones de ToF para el Sujeto 1. Se evidencia que en condiciones NLOS (gráficas inferiores o colas largas), la dispersión temporal es mucho mayor, lo que implica una incertidumbre directa en la estimación de distancia.

\begin{figure}[H]
    \centering
    \begin{minipage}[b]{0.48\columnwidth}
        \includegraphics[width=\linewidth]{pecho_sujeto1_ancla1_tof_sin_outliers.png}
        \caption*{Ancla 1}
    \end{minipage}
    \hfill
    \begin{minipage}[b]{0.48\columnwidth}
        \includegraphics[width=\linewidth]{pecho_sujeto1_ancla2_tof_sin_outliers.png}
        \caption*{Ancla 2}
    \end{minipage}
    \caption{Distribuciones de ToF (Pecho, Sujeto 1). La mayor dispersión en ciertos enlaces indica obstrucción severa.}
    \label{fig:tof_pecho_dist}
\end{figure}

\subsection{Caracterización Probabilística del Error}
El modelado estadístico del error es esencial para el diseño de algoritmos de navegación robustos. Mediante el Criterio de Información de Akaike (AIC), se evaluó el ajuste de diversas distribuciones teóricas (Gamma, Log-normal, Exponencial, Normal) a los datos empíricos, seleccionando el modelo que mejor capturara la estructura de colas del error.

En condiciones LOS, la distribución Gamma resultó ser el modelo óptimo (ver Figura \ref{fig:ajuste_los}), caracterizando adecuadamente la asimetría positiva leve inherente a la medición de ToF en propagación directa. Sin embargo, en condiciones NLOS, la estructura del error cambia radicalmente. La Figura \ref{fig:ajuste_nlos} demuestra de forma concluyente que \textbf{la distribución Log-normal ofrece el mejor ajuste de los datos NLOS}, superando significativamente a la Gamma según criterio AIC. Este hallazgo no es casual: refleja la naturaleza multiplicativa del ruido en entornos obstruidos. A diferencia de los errores Gaussianos aditivos que asumen los filtros lineales clásicos, los errores NLOS resultan de procesos multiplicativos donde múltiples factores (atenuación por BS, retardo por multipath, distorsión por difracción) se concatenan, generando distribuciones con colas pesadas característica de procesos multiplicativos. El error de distancia en NLOS puede expresarse aproximadamente como: $d_{\text{error}} \propto d_{\text{real}} \cdot (1 + \alpha_1 \cdot \alpha_2 \cdot \alpha_3)$, donde $\alpha_i$ representan factores multiplicativos de atenuación, cuya composición genera la distribución Log-normal observada.

\begin{figure}[H]
    \centering
    \begin{minipage}[b]{0.48\columnwidth}
        \includegraphics[width=\linewidth]{ajuste_hist_Datos_LOS.png}
        \caption{Ajuste LOS: Gamma óptimo.}
        \label{fig:ajuste_los}
    \end{minipage}
    \hfill
    \begin{minipage}[b]{0.48\columnwidth}
        \includegraphics[width=\linewidth]{ajuste_hist_Datos_NLOS.png}
        \caption{Ajuste NLOS: Log-normal óptimo.}
        \label{fig:ajuste_nlos}
    \end{minipage}
    \caption{Comparación de ajustes de distribución de probabilidad para errores de distancia.}
\end{figure}

La desviación respecto a la normalidad es crítica para comprender las limitaciones de los filtros tradicionales. La Figura \ref{fig:qq_nlos} presenta el gráfico Q-Q para datos NLOS, donde la divergencia extrema en la cola superior confirma que los errores grandes son mucho más frecuentes de lo que predeciría una distribución Gaussiana (test de Shapiro-Wilk: $p \ll 0.001$). Específicamente, la curvatura pronunciada en el extremo derecho indica que la probabilidad de ``eventos de error grande'' en NLOS es órdenes de magnitud mayor que bajo la hipótesis Gaussiana, explicando por qué filtros lineales clásicos (Kalman lineal) son ineficaces en presencia de obstrucción corporal severa. El filtro extendido de Kalman (EKF) empleado en Fase 2, al ser informado por estas desviaciones estándar empíricas (Tabla \ref{tab:tof_stats}) en la configuración de su matriz de covarianza de medición, proporciona un mecanismo de adaptación que, aunque no resuelve completamente la no-Gaussianidad, mejora significativamente la coherencia de trayectoria respecto a un filtro lineal no adaptativo.

\begin{figure}[H]
    \centering
    \includegraphics[width=0.7\columnwidth]{ajuste_qq_Datos_NLOS.png}
    \caption{Gráfico Q-Q (Datos NLOS vs. Normal Teórica). La curvatura indica una fuerte desviación de la gaussianidad (línea roja).}
    \label{fig:qq_nlos}
\end{figure}

\subsection{Evaluación del Sistema de Posicionamiento 2D}
Para validar el impacto de los errores de distancia en una aplicación real, se implementó un sistema de posicionamiento 2D en un salón de 7.4 $\times$ 10.4 m, equipado con cuatro anclas fijas en las esquinas (coordenadas: (10.11, 0.37), (0.28, 0.38), (0.28, 6.72), (10.09, 6.97) m). Se trazó una trayectoria de prueba compuesta por 18 puntos de parada (detención estacionaria de ~70 segundos en cada punto) distribuidos en una cuadrícula de 3 filas $\times$ 6 columnas con espaciamiento de 1.5~m.

\subsubsection{Caso de Referencia: Tag sin Cuerpo Humano}
La Figura \ref{fig:kalman_tag} muestra los resultados para el \textit{tag} de referencia (sin cuerpo humano). El Filtro de Kalman Extendido (línea azul) suaviza eficazmente las fluctuaciones de la multilateración cruda (puntos naranjas), logrando una trayectoria que se ajusta fielmente a la referencia real (puntos verdes), con un error de posicionamiento promedio de 48 cm. Este desempeño representa el caso óptimo del sistema: cuando no existe obstrucción corporal, el ruido de medición es aproximadamente gaussiano de media cero, y el EKF converge correctamente a las posiciones reales.

\begin{figure}[H]
    \centering
    \includegraphics[width=0.9\linewidth]{kalman_tag.pdf}
    \caption{Trayectoria estimada para el tag de referencia: comparación entre posiciones reales y estimadas.}
    \label{fig:kalman_tag}
\end{figure}

\subsubsection{Caso con Obstrucción Corporal: Cadera}
Al introducir el factor humano con el tag en la cadera, la degradación es drástica. La Figura \ref{fig:kalman_cadera} ilustra este caso crítico. Se aprecian desviaciones significativas en múltiples puntos de la trayectoria, con errores de posición que alcanzan hasta 95 cm. Aunque el filtro de Kalman (línea azul) logra suavizar las fluctuaciones aleatorias (reduciendo el jitter en comparación con los puntos naranjas crudos de multilateración), la línea filtrada converge consistentemente LEJOS de la posición verdadera (puntos verdes).

Esta separación sistemática entre la trayectoria suavizada y la referencia real no es noise—es BIAS. El problema fundamental es que el EKF asume ruido de medición con media cero: $\mathbf{v}_k \sim \mathcal{N}(0, \mathbf{R})$. Sin embargo, bajo obstrucción corporal severa, la obstrucción introduce un sesgo positivo sistemático: cada distancia medida es SIEMPRE mayor que la distancia real $d_{\text{medida}} = d_{\text{real}} + \text{bias}(\text{ubicación})$. Como el filtro carece de una referencia externa para estimar este sesgo en tiempo real, converge hacia la medición errónea suavizada, amplificando el error posicional. En otras palabras: el EKF eficientemente suaviza una medición incorrecta.

\begin{figure}[H]
    \centering
    \includegraphics[width=0.9\linewidth]{kalman_cadera_sujeto1.pdf}
    \caption{Trayectoria estimada para la ubicación en Cadera (Sujeto 1). El filtro suaviza el jitter (línea azul vs puntos naranjas) pero no corrige el sesgo sistemático, resultando en desviaciones consistentes de la referencia verdadera (puntos verdes).}
    \label{fig:kalman_cadera}
\end{figure}

\subsubsection{Comparación: Pecho}
Similares efectos se observan en la ubicación del pecho (Figura \ref{fig:kalman_pecho}), donde el error osciló entre 70 y 91 cm. El sesgo es ligeramente menor que en cadera pero igualmente presente.

\begin{figure}[H]
    \centering
    \includegraphics[width=0.9\linewidth]{kalman_pecho_sujeto1.pdf}
    \caption{Trayectoria estimada para la ubicación en Pecho (Sujeto 1). Similar a la cadera, el EKF suaviza pero no elimina el sesgo.}
    \label{fig:kalman_pecho}
\end{figure}

