La tecnología de Banda Ultra Ancha (UWB) se ha consolidado como una solución prometedora para Sistemas de Posicionamiento en Interiores (IPS) por su alta precisión. Sin embargo, su desempeño en aplicaciones de seguimiento de personas se ve afectado por el fenómeno de obstrucción corporal (\textit{Body Shadowing}, BS), que degrada la exactitud de la localización. Este artículo presenta una investigación experimental sistemática del efecto de la BS en sistemas IPS UWB operando en la banda de 6.5 GHz, una frecuencia menos explorada que las bandas inferiores.

La metodología se basó en un sistema experimental con cuatro anclas fijas y un nodo móvil portado por participantes en siete ubicaciones corporales (cabeza, pecho, cadera, muñeca, tobillo, mano, rodilla). Se recolectaron mediciones de Tiempo de Vuelo (ToF) en condiciones de Línea de Vista (LOS) y Sin Línea de Vista (NLOS) en escenarios de interior y exterior.

Los resultados revelan que la ubicación del dispositivo es crítica. La cabeza ofrece el desempeño más consistente, con un error absoluto medio (MAE) de 4.87 cm en LOS exterior, degradándose de forma controlada a 18.16 cm en NLOS interior. Por el contrario, la cadera presenta la mayor degradación, pasando de un MAE de 7.22 cm en LOS a 95.11 cm en NLOS. El análisis estadístico demostró que los errores en NLOS no siguen una distribución gaussiana, lo que requiere algoritmos de localización robustos. El sistema de posicionamiento 2D alcanzó una exactitud de entre 55-95 cm dependiendo de la ubicación, demostrando la viabilidad de la tecnología para múltiples aplicaciones prácticas.
\textbf{Palabras clave:} Banda Ultra Ancha (UWB), Sistemas de Posicionamiento en Interiores (IPS), Obstrucción Corporal, Tiempo de Vuelo (ToF), Exactitud de Localización, 6.5 GHz.
